% =============================================================================
% FILE: /sections/05_policy.tex
% PURPOSE: To answer the "So what?" question. Connects the abstract model
% to real-world applications and provides actionable recommendations.
% Written by the Policy & Implications team.
% =============================================================================
\section{Policy Implications and Applications}\label{sec:policy}

Our theoretical and empirical results carry significant implications for how policymakers, program managers, and social enterprises should think about designing and evaluating interventions. The core message is that **expanding access to opportunities can be as important as improving the quality of the opportunities themselves.**

\subsection{Reinterpreting Program Evaluations}
A key contribution of our framework is that it provides a new lens through which to interpret the results of existing program evaluations. Many studies conclude with "no significant effect," leading funders to defund programs that may in fact be highly effective for those who can access them.

\subsubsection{Case Study: University Mentorship Programs}
Consider a typical university mentorship program for first-generation students. An evaluation might find that, on average, the program has no effect on GPA. Our model suggests an alternative interpretation: it's possible that only the most motivated and well-informed students were able to navigate the bureaucracy to sign up. The "no effect" finding could be the result of comparing these high-achieving participants to other high-achievers who didn't participate. The true effect for a less-informed student, had they gained access, might be very large. The policy recommendation is not to cancel the program, but to fix the outreach and enrollment process.

\subsection{Recommendations for Program and Platform Design}
Our model suggests a shift in focus for organizations that provide opportunities, from merely offering a service to actively managing and expanding the choice sets of their target audience.

\subsubsection{Measure Access, Not Just Outcomes}
Organizations should track "access metrics" in addition to outcome metrics. This could include simple surveys like, "Before interacting with us, were you aware that this opportunity existed?" Comparing awareness across different demographic groups can provide a direct measure of the "invisible sets" problem.

\subsubsection{Design for Discovery}
Platforms should be designed not as static repositories of information but as active discovery engines. This includes:
\begin{itemize}
    \item **Investing in outreach:** Go to where potential users are, rather than expecting them to come to you.
    \item **Using network effects for good:** Encourage peer-to-peer referrals but also build bridges to connect disparate social clusters.
    \item **Centralized clearinghouses:** For high-stakes opportunities like internships, centralized application systems (like the medical residency match) can dramatically reduce the influence of social capital and expand the choice set for everyone.
\end{itemize}

\subsection{Applications in Education and Labor Markets}
The "invisible sets" framework has direct applications in two critical areas.

\subsubsection{Educational Opportunity}
In education, access to Advanced Placement (AP) courses, gifted programs, and college application guidance is often non-random. Our model suggests that policies of "automatic enrollment" or universal screening for these programs are effective precisely because they replace a latent, unequal choice set with a universal, observed one.

\subsubsection{Labor Market Entry}
In labor markets, referral networks are a dominant source of hiring, creating a classic invisible sets problem. Policies that encourage or mandate the public posting of all job openings serve to make choice sets more transparent and equitable. Similarly, mentorship platforms that connect students with professionals outside their existing networks are, in essence, a tool for choice set expansion.