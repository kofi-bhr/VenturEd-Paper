% =============================================================================
% FILE: /sections/03_theory.tex
% PURPOSE: To explain our theoretical model and results.
% Written by the Theoretical Modeling team.
% NOTE: The formal math (definitions, theorems) is in /model/math_framework.tex.
% This file IMPORTS that math and provides the intuitive, plain-English
% explanations that make it understandable.
% =============================================================================
\section{Theoretical Framework}\label{sec:theory}

% --- IMPORT THE FORMAL MODEL SETUP ---
% This command inserts the content from the math_framework.tex file.
% =============================================================================
% FILE: /model/math_framework.tex
% PURPOSE: Defines the core mathematical model and states the main theorems.
% This file is imported into sections/03_theory.tex.
% =============================================================================

\section{The Model}\label{sec:model}

\subsection{Setup: Environment and Agents}

% The goal here is to formally define the world we are analyzing.
% Each agent has characteristics, a set of choices, preferences, and outcomes.

\begin{definition}[Choice Environment]
We model a population of individuals indexed by $i$. Each individual is characterized by a tuple $(Z_i, g_i, C_i, u_i(\cdot), \{y_i(x)\}_{x \in \X})$:
\begin{itemize}
    \item Observable characteristics $Z_i \in \Z$.
    \item Group membership $g_i \in \G$.
    \item A latent (unobserved by the researcher) \textbf{choice set} $C_i \subseteq \X$, where $\X$ is the universe of all possible choices (e.g., programs, jobs).
    \item A utility function $u_i: \X \to \R$.
    \item A set of potential outcomes $\{y_i(x)\}_{x \in \X}$, where $y_i(x)$ is the outcome individual $i$ would realize if they were to choose option $x$.
\end{itemize}
\end{definition}

% Explain how individuals make decisions. We assume rationality.

\begin{definition}[Choice Behavior]
Individual $i$ observes their choice set $C_i$ and chooses the option $x_i \in C_i$ that maximizes their utility:
\begin{equation}
x_i = \arg\max_{x \in C_i} u_i(x)
\end{equation}
The researcher observes the realized choice $x_i$ and the realized outcome $y_i = y_i(x_i)$. The researcher does not observe $C_i$ or the potential outcomes for choices not taken.
\end{definition}

% State the key assumption that drives our results: access is not random.

\begin{assumption}[Choice Set Formation]
The distribution of choice sets depends on an individual's observable characteristics.
\begin{equation}
C_i \sim F_C(\cdot | Z_i, g_i)
\end{equation}
This allows for the possibility that different groups have systematically different access to opportunities.
\end{assumption}

\subsection{Researcher's Problem: Identification}

% Clearly define the causal parameter we want to learn.

For a binary treatment $T \in \{0, 1\}$, where $x=1$ is treatment and $x=0$ is control, the researcher's goal is to learn the Average Treatment Effect (ATE):
\begin{equation}
\tau \equiv \E[y_i(1) - y_i(0)]
\end{equation}

% Define what "identification" means. This is a key technical concept.

\begin{definition}[Identification]
A parameter (like $\tau$) is \textbf{point identified} if the observable data distribution is consistent with only one possible value of the parameter. It is \textbf{partially identified} if the data restricts the parameter to a specific range (bounds).
\end{definition}

% =============================================================================
% --- THEORETICAL RESULTS ---
% State each theorem here. The proofs go in the appendix.
% =============================================================================

\subsection{Main Theoretical Results}

\begin{theorem}[Non-Identification of Treatment Effects]\label{thm:non_id}
When choice sets $C_i$ are latent and may vary across individuals based on their characteristics, the Average Treatment Effect $\tau$ is not point identified, even with infinite data.
\end{theorem}

\begin{theorem}[Partial Identification via Bounds]\label{thm:bounds}
Let $p(z, g) \equiv \Prob(1 \in C_i | Z_i=z, g_i=g)$ be the probability that an individual in group $(z,g)$ has access to treatment. If we make the following assumptions:
\begin{enumerate}[(i)]
    \item \textbf{Support Condition:} For each group $(z,g)$, we know the true access probability lies in a range $[\underline{p}_{zg}, \overline{p}_{zg}]$.
    \item \textbf{Monotonicity:} For all individuals, $y_i(1) \ge y_i(0)$.
\end{enumerate}
Then, the ATE is partially identified and contained within a set of values $[\tau_L, \tau_U]$ that can be calculated from the observable data and the assumption bounds.
\end{theorem}

\begin{theorem}[Network Homophily and Identification]\label{thm:networks}
If access to opportunities (i.e., the contents of $C_i$) is spread through a social network $\N$, and the network exhibits homophily (similar people are connected), then both allocative efficiency and the researcher's ability to identify treatment effects are reduced.
\end{theorem}

% Now, provide the intuitive walkthrough for the model that was just imported.

\subsection{Intuition Behind the Model}
The model presented in Section \ref{sec:model} formalizes the simple idea of the "lunch menu" problem discussed in the introduction. We define individuals by their observable traits ($Z_i, g_i$), their hidden menu of options ($C_i$), and their preferences ($u_i(\cdot)$). The key element is that the researcher cannot see $C_i$. This single assumption—that choice sets are latent—is the engine that drives all of our results. By assuming that the availability of these choice sets can depend on an individual's group (Assumption 1), we build a world where access to opportunity can be systematically unequal.

\subsection{Explaining the Main Results}
Our model generates three main theoretical results, which we state formally as Theorems \ref{thm:non_id} through \ref{thm:networks}. Here, we explain the intuition behind each.

\subsubsection{Why the Treatment Effect is Not Identified (Theorem \ref{thm:non_id})}
Theorem \ref{thm:non_id} is a negative but crucial result. It states that if we do not know who could have accessed a program, we can never know for sure if the program worked. The proof provides a stark example: imagine two groups of people who all enrolled in a tutoring program. In one world, they all had other options but chose tutoring because it was the most effective. In another world, they had no other options and were forced into it. The observed data on enrollment and outcomes could look identical in both worlds, but the true effectiveness of the program would be completely different. Because we cannot distinguish between these worlds from the data alone, the effect is not "point identified."

\subsubsection{What We Can Still Learn: Partial Identification (Theorem \ref{thm:bounds})}
While we cannot pin down a single number for the treatment effect, we are not left with nothing. Theorem \ref{thm:bounds} shows that we can derive a "range of possibilities" for the true effect. The logic is to ask: "Given what we observe, what is the most optimistic and most pessimistic scenario consistent with the data?" For example, the pessimistic scenario might assume that all the high-performing individuals who took the treatment would have done just as well without it. The optimistic scenario might assume the opposite. The true effect must lie somewhere between these two extremes. The width of this range, or the "bounds," becomes a measure of our fundamental uncertainty.

\subsubsection{How Social Networks Make Things Worse (Theorem \ref{thm:networks})}
Theorem \ref{thm:networks} connects our model to the real world of referrals and social connections. It makes two points. First, when opportunities spread through segregated social networks ("homophily"), it destroys the useful variation a researcher needs. Instead of having some treated and some untreated people within the same group to compare, you might have one group where everyone has access and another where no one does. This makes it almost impossible to separate the effect of the treatment from the underlying differences between the groups. Second, this is inefficient. An opportunity for a talented student might never reach them simply because it gets "stuck" in a different social circle, leading to a loss for both the student and society.