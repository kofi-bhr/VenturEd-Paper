% =============================================================================
% FILE: /sections/04_simulation.tex
% PURPOSE: To demonstrate the theoretical results using simulated data.
% Written by the Empirical Analysis team.
% The goal is to show, not just tell. We will show that naive methods fail
% and that our framework provides a useful way to think about the problem.
% =============================================================================
\section{Simulation Analysis}\label{sec:simulation}

To illustrate the consequences of our theoretical findings, we conduct a series of Monte Carlo simulations. These simulations allow us to create an idealized world where we know the true treatment effect and the true choice sets, and then observe how different estimation strategies perform when the choice sets are hidden from the researcher.

\subsection{Simulation Design: The Data Generating Process (DGP)}
% Describe the setup of your simulation here.
% Be specific about the parameters.
We simulate a population of $N=1000$ individuals, divided into two equally sized groups, $g \in \{A, B\}$. An individual's potential outcome without treatment is $y_i(0) \sim N(0, 1)$. The true treatment effect is $\tau=0.5$, so their potential outcome with treatment is $y_i(1) = y_i(0) + 0.5$.

The key mechanism is the choice set formation. We stipulate that individuals in Group A always have access to the treatment ($1 \in C_i$ for all $i \in A$), while individuals in Group B only have access with probability $p_B$. We will vary $p_B$ from 0 to 1 to simulate different levels of access inequality.

\subsection{Result 1: The Failure of Naive Estimation}
% Present the first main finding from the simulations.
Our first goal is to show that a naive Ordinary Least Squares (OLS) regression of outcomes on treatment status yields a biased estimate of the true treatment effect.
% Explain the experiment and the result.
We run our simulation 5,000 times for a fixed level of access inequality ($p_B=0.2$). As shown in Figure \ref{fig:ols_bias}, the distribution of the OLS estimates is centered far from the true effect of 0.5. The naive estimator is systematically biased because it wrongly attributes the higher average outcomes of Group A (who have better access) to the treatment itself, rather than to selection.

% Placeholder for the first figure.
\begin{figure}[h!]
    \centering
    % The actual image file will go in the /figures/ folder.
    \includegraphics[width=0.7\textwidth]{figures/ols_bias_distribution.png}
    \caption{Distribution of Naive OLS Estimates vs. True Effect}
    \label{fig:ols_bias}
    \Description{A histogram showing the distribution of estimated treatment effects from a naive OLS regression across 5,000 simulations. A vertical line shows the true effect, and the histogram is centered far away from it, demonstrating bias.}
\end{figure}

\subsection{Result 2: The Performance of Bounds}
% Present the second main finding.
Next, we show that the partial identification framework correctly characterizes our uncertainty. We calculate the theoretical bounds on the ATE for each simulation.
% Explain the result.
Figure \ref{fig:bounds_coverage} plots the calculated bounds for 100 different simulation runs. The horizontal line represents the true ATE. In over 95\% of cases, the calculated interval (the vertical bars) successfully contains the true effect. This demonstrates that while we cannot get a single point estimate, the bounds provide a reliable range for the true parameter. The width of the bounds shrinks as access inequality decreases (as $p_B \to 1$).

% Placeholder for the second figure.
\begin{figure}[h!]
    \centering
    \includegraphics[width=0.7\textwidth]{figures/bounds_coverage.png}
    \caption{Coverage of Partial Identification Bounds}
    \label{fig:bounds_coverage}
    \Description{A plot showing the calculated bounds (as vertical lines) for 100 simulations. A horizontal line shows the true treatment effect. Most of the vertical lines cross the horizontal line, indicating that the bounds successfully contain the true value.}
\end{figure}

\subsection{Result 3: Simulating a Policy Intervention}
% Show how our framework can be used for policy analysis.
Finally, we simulate a policy intervention. We imagine the government can spend a budget to increase the probability of access for Group B from $p_B=0.2$ to $p_B=0.5$. Our framework allows us to analyze the effect of this policy not just on outcomes, but on identification. As shown in Table \ref{tab:policy_sim}, the intervention not only improves average outcomes but also dramatically shrinks the width of the identification bounds, increasing the precision of our knowledge about the program's effectiveness.

% Placeholder for a table.
\begin{table}[h!]
    \centering
    \caption{Effect of a Simulated Policy Intervention}
    \label{tab:policy_sim}
    \begin{tabular}{@{}lcc@{}}
    \toprule
    Metric                & Pre-Intervention ($p_B=0.2$) & Post-Intervention ($p_B=0.5$) \\ \midrule
    Average ATE Bound Width & 0.45                         & 0.22                          \\
    Average Outcome       & 0.15                         & 0.30                          \\ \bottomrule
    \end{tabular}
\end{table}