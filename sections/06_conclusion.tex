% =============================================================================
% FILE: /sections/06_conclusion.tex
% PURPOSE: To summarize our contributions and look to the future.
% Written by the Policy & Implications team.
% Keep it concise and impactful.
% =============================================================================
\section{Conclusion}\label{sec:conclusion}

% --- PARAGRAPH 1: SUMMARIZE THE CORE CONTRIBUTION ---
% In one paragraph, restate the problem, our approach, and our main findings.
In this paper, we have developed a formal economic model to understand the consequences of "invisible sets," where individuals face unobserved and unequal access to opportunities. We have shown that this common real-world phenomenon poses a fundamental challenge to causal inference, rendering treatment effects unidentified. By applying the tools of partial identification, we characterized what can and cannot be learned in such environments and demonstrated how social network structures can exacerbate these problems.

% --- PARAGRAPH 2: DISCUSS LIMITATIONS ---
% Acknowledge the limitations of the model. This shows intellectual honesty.
Our model, like any, is a simplification of reality. We have relied on assumptions such as monotonicity and have primarily focused on a binary treatment case. Furthermore, while our framework provides bounds on treatment effects, estimating these bounds with real-world data requires its own set of methodological choices and can be challenging in practice. Our work should be seen as a theoretical foundation upon which more complex empirical applications can be built.

% --- PARAGRAPH 3: SUGGEST FUTURE RESEARCH DIRECTIONS ---
% End on a forward-looking note. What are the next interesting questions?
This research opens several avenues for future work. Empirically, there is a need to develop and validate survey instruments that can effectively measure latent choice sets in different contexts. Theoretically, our model could be extended to a dynamic setting to understand how choice sets evolve over an individual's lifetime. Finally, our policy simulations suggest a rich agenda for field experiments that directly test the effectiveness of different "choice set expansion" strategies, such as information interventions versus network-bridging programs. Ultimately, our framework suggests that understanding what people *can* choose is as important as understanding what they *do* choose.