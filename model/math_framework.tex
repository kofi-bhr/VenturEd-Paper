% =============================================================================
% FILE: /model/math_framework.tex
% PURPOSE: Defines the core mathematical model and states the main theorems.
% This file is imported into sections/03_theory.tex.
% =============================================================================

\section{The Model}\label{sec:model}

\subsection{Setup: Environment and Agents}

% The goal here is to formally define the world we are analyzing.
% Each agent has characteristics, a set of choices, preferences, and outcomes.

\begin{definition}[Choice Environment]
We model a population of individuals indexed by $i$. Each individual is characterized by a tuple $(Z_i, g_i, C_i, u_i(\cdot), \{y_i(x)\}_{x \in \X})$:
\begin{itemize}
    \item Observable characteristics $Z_i \in \Z$.
    \item Group membership $g_i \in \G$.
    \item A latent (unobserved by the researcher) \textbf{choice set} $C_i \subseteq \X$, where $\X$ is the universe of all possible choices (e.g., programs, jobs).
    \item A utility function $u_i: \X \to \R$.
    \item A set of potential outcomes $\{y_i(x)\}_{x \in \X}$, where $y_i(x)$ is the outcome individual $i$ would realize if they were to choose option $x$.
\end{itemize}
\end{definition}

% Explain how individuals make decisions. We assume rationality.

\begin{definition}[Choice Behavior]
Individual $i$ observes their choice set $C_i$ and chooses the option $x_i \in C_i$ that maximizes their utility:
\begin{equation}
x_i = \arg\max_{x \in C_i} u_i(x)
\end{equation}
The researcher observes the realized choice $x_i$ and the realized outcome $y_i = y_i(x_i)$. The researcher does not observe $C_i$ or the potential outcomes for choices not taken.
\end{definition}

% State the key assumption that drives our results: access is not random.

\begin{assumption}[Choice Set Formation]
The distribution of choice sets depends on an individual's observable characteristics.
\begin{equation}
C_i \sim F_C(\cdot | Z_i, g_i)
\end{equation}
This allows for the possibility that different groups have systematically different access to opportunities.
\end{assumption}

\subsection{Researcher's Problem: Identification}

% Clearly define the causal parameter we want to learn.

For a binary treatment $T \in \{0, 1\}$, where $x=1$ is treatment and $x=0$ is control, the researcher's goal is to learn the Average Treatment Effect (ATE):
\begin{equation}
\tau \equiv \E[y_i(1) - y_i(0)]
\end{equation}

% Define what "identification" means. This is a key technical concept.

\begin{definition}[Identification]
A parameter (like $\tau$) is \textbf{point identified} if the observable data distribution is consistent with only one possible value of the parameter. It is \textbf{partially identified} if the data restricts the parameter to a specific range (bounds).
\end{definition}

% =============================================================================
% --- THEORETICAL RESULTS ---
% State each theorem here. The proofs go in the appendix.
% =============================================================================

\subsection{Main Theoretical Results}

\begin{theorem}[Non-Identification of Treatment Effects]\label{thm:non_id}
When choice sets $C_i$ are latent and may vary across individuals based on their characteristics, the Average Treatment Effect $\tau$ is not point identified, even with infinite data.
\end{theorem}

\begin{theorem}[Partial Identification via Bounds]\label{thm:bounds}
Let $p(z, g) \equiv \Prob(1 \in C_i | Z_i=z, g_i=g)$ be the probability that an individual in group $(z,g)$ has access to treatment. If we make the following assumptions:
\begin{enumerate}[(i)]
    \item \textbf{Support Condition:} For each group $(z,g)$, we know the true access probability lies in a range $[\underline{p}_{zg}, \overline{p}_{zg}]$.
    \item \textbf{Monotonicity:} For all individuals, $y_i(1) \ge y_i(0)$.
\end{enumerate}
Then, the ATE is partially identified and contained within a set of values $[\tau_L, \tau_U]$ that can be calculated from the observable data and the assumption bounds.
\end{theorem}

\begin{theorem}[Network Homophily and Identification]\label{thm:networks}
If access to opportunities (i.e., the contents of $C_i$) is spread through a social network $\N$, and the network exhibits homophily (similar people are connected), then both allocative efficiency and the researcher's ability to identify treatment effects are reduced.
\end{theorem}